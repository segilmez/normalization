\documentclass{article}
\usepackage{graphicx}
\usepackage{color}
\usepackage{comment}
\usepackage{amssymb}
\usepackage{amsthm}

\newtheorem{definition}{Definition}
\newtheorem{property}{Property}
\newtheorem{proposition}{Proposition}
\newtheorem{conjecture}{Conjecture}
\newtheorem{example}{Example}
\newtheorem{notation}{Notation}
\newtheorem{theorem}{Theorem}

%%%%%%%%%%%%%%%%%%%%%%%%%%%%%%%%%%%%%%%%%%%%%%%%%%%%%%%%%%%%%%%%%%%%%

% Algorithms and pseudo code
\usepackage{verbatim}
\usepackage{algorithm}
\usepackage{algorithmicx}
\usepackage{algpseudocode}

% Graphics and display
\usepackage{float}
\usepackage{graphicx}
\usepackage{subfig}
\usepackage{enumerate}
\usepackage{url}
\usepackage{multirow}

% Math symbols and environments
\usepackage{amsmath}
\usepackage{amssymb}
\usepackage{stmaryrd} % \varcurlyvee


\usepackage{calc}%#%
\usepackage{enumitem} %%% used in framed graph

\date{ }
% TikZ

\usepackage{tikz}
\usetikzlibrary{shapes,arrows}
\usetikzlibrary{positioning}
\usetikzlibrary{calc}

\usepackage{tikz-3dplot}

\definecolor{myyellow}{RGB}{255,255,150}
\definecolor{mylavender}{RGB}{125,249,255}
\definecolor{mygreen}{RGB}{144,238,14}
\definecolor{myred}{RGB}{255,0,0}

\newcommand\mytext[3][\scriptsize]{#2\\#1 #3}
\newcommand\mynode[4][]{%
  \node[mynode,#1,text width=\the\dimexpr#2cm] (#3) {\mytext{#3}{#4}}; 
}
\newcommand\mynot[4][]{%
  \node[mynot,#1,text width=\the\dimexpr#2cm] (#3) {\mytext{#3}{#4}}; 
}

\setcounter{secnumdepth}{4}

%%%%%%%%%%%%%%%%
%%%% Macros %%%%
%%%%%%%%%%%%%%%%

%%% Math
\newcommand{\nat}{\mathbb{N}}   % Natural numbers
\newcommand{\rat}{\mathbb{Q}}   % Rational numbers
\newcommand{\real}{\mathbb{R}}  % Real numbers
\newcommand{\runit}{[0, 1]}    % The real unit interval


% = with "hip." on the top, useful for indicating where a hypothesis comes in
\newcommand{\heq}{\stackrel{\text{\fontsize{3pt}{3pt}\selectfont hip.}}{=}}
% = because of the de Morgan laws.
\newcommand{\dmeq}{\stackrel{\text{\tiny{dM}}}{=}}


%%% Sets
\newcommand{\args}{\mathcal{A}} % Set of all arguments
\newcommand{\att}{\mathcal{R}}  % Set of all attacks
\newcommand{\valueset}{L}

\newcommand{\obj}{\mathcal{O}} % Set of all arguments and attacks as a union

%%% Votes on arguments
\newcommand{\varg}{V_{\args}}   % Function giving votes on arguments
\newcommand{\vargpro}[1]{\varg^+\left(#1\right)} % Pro votes on arguments
\newcommand{\vargcon}[1]{\varg^-\left(#1\right)} % Con votes on arguments
\newcommand{\vargtot}[1]{\varg^{max}\left(#1\right)} % Max votes on arguments

%%% Votes on attacks
\newcommand{\vatt}{V_{\att}}   % Function giving votes on attacks
\newcommand{\vattpro}[1]{\vatt^+\left(#1\right)} % Pro votes on attacks
\newcommand{\vattcon}[1]{\vatt^-\left(#1\right)} % Con votes on attacks

%%% Attack relations
% Attackers of a given argument
\newcommand{\attackers}[1]{\att^\text{-}\left(#1\right)} 
% Attackers of a given argument for the alternative framework F'
\newcommand{\altattackers}[1]{\att^{\prime\text{-}}\left(#1\right)}
% Ancestors of given argument according to the attack relation
\newcommand{\ancestors}[1]{\att^*\left(#1\right)} 

%%% Frameworks
\newcommand{\safid}{F}               % A single SAF, given by identifier
\newcommand{\safset}{\mathcal{F}}    % Set of all SAFs

\newcommand{\saf}{\safid = \safbody} % Framework id and respective tuple
\newcommand{\safbody}{\langle \args, \att, \varg, \vatt \rangle} % SAF tuple
\newcommand{\oldsaf}{\safid = \oldsafbody} % Ex Framework id and respective tuple
\newcommand{\oldsafbody}{\langle \args, \att, V \rangle} % old SAF tuple
% Alternative framework, same as \safbody but with ' everywhere ;)
\newcommand{\altsafbody}{\langle \args', \att', \varg', \vatt' \rangle} 

\newcommand{\safbodyO}{\langle \args, \att, \obj, V \rangle} % SAF tuple with objects
\newcommand{\safO}{\safid = \safbodyO} % Framework id and respective tuple with objects

%%% Semantics
\newcommand{\semid}{\mathcal{S}}        % Semantic framework identifier
% Semantic framework tuple
\newcommand{\sembody}{\left\langle \valueset,\SAFand_1, \SAFand_2,\SAFor,\lnot,\tau \right\rangle}
\newcommand{\semdef}{\semid = \sembody}     % Semantic framework id and tuple
\newcommand{\semprod}[1]{\semid^\cdot_{#1}} % Product semantic framework
\newcommand{\semsub}{\semid^\text{-}}       % Subtraction semantic framework
\newcommand{\semmax}{\semid^\text{max}}     % Max semantic framework

\newcommand{\sembodyNew}{\left\langle \valueset,\SAFand_\mathcal{A}, \SAFand_\mathcal{R},\SAFor,\lnot,\tau \right\rangle} %New semantic body
\newcommand{\sembodyNewE}{\left\langle \valueset,\SAFand_\mathcal{A}, \SAFand_\mathcal{R},\SAFor,\lnot,\tau_{e} \right\rangle} %New semantic body

\newcommand{\SAFand}{\curlywedge}     % Logical and for SAF equations 
\newcommand{\SAFor}{\curlyvee}        % Logical or for SAF equations
\DeclareMathOperator*{\SAFOr}{\bigcurlyvee} % Big or notation, works as \sum
                             %\varcurlyvee also works, but is smaller
\DeclareMathOperator*{\SAFAnd}{\bigcurlywedge} % Big and notation, works as \sum

\newcommand{\modelset}{\mathcal{M}}   % Set of all models


%#% old commands
\newcommand{\afit}{\textit{AF}}
\newcommand{\af}{\afit = \langle \args, \att \rangle}
\newcommand{\vote}{V}
\newcommand{\sem}{\mathcal{S}}

\newcommand{\ssv}{\mathcal{V}}
\newcommand{\tv}{\mathcal{T}}
\newcommand{\pv}{\mathcal{P}}
\newcommand{\xv}{X}
\newcommand{\ev}{\mathcal{E}}

\newcommand{\safit}{F}

\newcommand{\tupd}{\curlywedge}
\newcommand{\tatt}{\curlyvee}
\newcommand{\Tatt}{\varcurlyvee}

\newcommand{\argarray}{\{x_1, ..., x_n\}}

\newcommand{\voteset}{\mathcal{V}}
\newcommand{\vpro}{\vote^+}
\newcommand{\vcon}{\vote^-}

%%% Mappings
\newcommand{\mapping}{\Phi}

%%% Macro for framed graph
\newlist{tikzitem}{itemize}{1}
\setlist[tikzitem,1]{label=$\bullet$,nolistsep,leftmargin=*}

%%% Normalization related stuff
\newcommand{\dataset}{\mathcal{D}}   % Data set
\newcommand{\clusterset}{\mathcal{C}}   % Cluster set
\newcommand{\ssset}{\mathcal{T}}   % Social support set

%%%%%%%%%%%%%%%%%%%%%%%%%%%%%%%%%%%%%%%%%%%%%%%%%%%%%%%%%%%%%%%%%%%%%
%%%%%%%%%%%%%%%%%%%%%%%%%%%%%%%%%%%%%%%%%%%%%%%%%%%%%%%%%%%%%%%%%%%%%

\begin{document}
\date{}
\title{Discussion on comments}
\maketitle

%\section{Discussion on comments}


%%%%%%%%%%%%%%%%%%%%%%%%%%%%
\begin{comment}

\begin{center}
\fbox{\includegraphics[scale=0.9]{./graphs/.png}}
\end{center}

\begin{itemize} 
\item {\color{blue}
}
\end{itemize}

\end{comment}
%%%%%%%%%%%%%%%%%%%%%%%%%%%%


\begin{center}
\fbox{\includegraphics[scale=0.9]{./graphs/1.png}}
\end{center}

\begin{itemize} 
\item {\color{blue}
 I understand that $\obj$ may seem verbose at the first sight. But please recall the discussions we had on the report regarding the new classes of vote aggregation functions, before last summer. In the report we had a bunch of functions defined on social voting. For every trivial operation, we had to define one function over $\args$ and another one over $\att$. Thus we had agreed on adding $\obj$ to the framework, in order to eliminate this triviality.

  Even though to a lesser extent, in the current work we still benefit from this notion. So for the time being I keep the definition as it is if you've no further objections. 
}
\end{itemize}
%%%%%%%%%%%%
\begin{center}
\fbox{\includegraphics[scale=0.9]{./graphs/2.png}}
\end{center}

\begin{itemize}
\item {\color{blue} V's definition is given in Definition 1, but you're right that it wasn't clear enough. Hopefully it reads well now.

As you state it's not used in this definition, but concrete functions from this class of functions (as in Definition 5) will be defined given a voting function. That's why I thought we should better include it here as well. But maybe I'm mistaken.
}
\end{itemize}
%%%%%%%%%%%%
\begin{center}
\fbox{\includegraphics[scale=0.9]{./graphs/3.png}}
\end{center}

\begin{itemize} 
\item {\color{blue} Well $\tau$ is a function, $\ssset$ is a multiset and we've used capital letters for set/multiset symbols up to now. Am I missing something?
}
\end{itemize}
%%%%%%%%%%%%
\begin{center}
\fbox{\includegraphics[scale=0.9]{./graphs/4.png}}
\end{center}

\begin{itemize} 
\item {\color{blue} I'm not sure what you have in mind by "maximizing/minimizing the constants". I had some preliminary idea on the topic, maybe we're thinking the same thing.

So I'm assuming if somehow we agree on a set of \emph{normalized data sets}, we may utilize them as our training set. We may optimize the parameters with them, and thus fix an interval with respect to normalized sets regarding the two concepts of interest, \emph{the range} and \emph{the distribution}. 

If what you meant was \emph{maximizing the lower bound} and \emph{minimizing the upper bound} with a similar process, then I suppose we're on the same page.
}
\end{itemize}
%%%%%%%%%%%%
\begin{center}
\fbox{\includegraphics[scale=0.9]{./graphs/5.png}}
\end{center}
\begin{center}
\fbox{\includegraphics[scale=0.9]{./graphs/7.png}}
\end{center}

\begin{itemize} 
\item {\color{blue} I completely agree with your comment regarding Property 2. Indeed this was the whole purpose of Subsection 2.3 i.e. stating that with respect to the context of interest, we may decide to relinquish some properties.

I understand your criticism on the section being too preliminary, which is surely the case. But once again my only goal for now is stating the aforementioned need. Once we're satisfied with the state of the characterization and the properties, I'll put more effort on the concrete contexts of interests and associated set of properties.
}
\end{itemize}
%%%%%%%%%%%%
\begin{center}
\fbox{\includegraphics[scale=0.9]{./graphs/6.png}}
\end{center}

\begin{itemize} 
\item {\color{blue} If you are questioning the clusters in the current formalization, then the answer is \emph{they're almost the same}. The only thing is that in set theory partitions are defined over sets. We've to define clusters over multisets so we've to do minor modifications to the generic partition definition. For instance when formalizing \emph{collective exhaustiveness} property you would use union of sets, however here we utilize the multiset summation.

On the other hand, if you were inquiring about how partitions differ from clusters with respect to the machine learning literature, then they might greatly differ. That's because as you know there is no single definition of clusters that's agreed upon. For instance some clustering methods use a fuzzy definition where a datapoint $d$ may belong to a cluster $X$ with degree $0.3$ and to a cluster $Y$ with $0.7$. There are also \emph{transparent  clustering} methods which are binary wrt. membership but let datapoints belong to multiple clusters, etc. 
}
\end{itemize}




%BIZIM PAPER CITATION
%\cite{eml2013esaf}

\end{document}